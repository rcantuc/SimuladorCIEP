%%%%%%%%%%%%%%%
%%%         %%%
%%% PRÓLOGO %%%
%%%         %%%
%%%%%%%%%%%%%%%
\ifdefined\maindoc\else
%%%%%%%%%%%%%%%%%%
%%%% PREAMBLE %%%%
%%%%%%%%%%%%%%%%%%
\ifdefined\maindoc
	\documentclass[acronimos,noportada,prologo]{CIEP/ciepnew}
\else
	\documentclass[acronimos,noportada]{../CIEP/ciepnew}
\fi

\title{Simulador Fiscal \acs{CIEP}}
\titlesize{\fontsize{30}{30}}

\subtitle{\bfseries Manual de usuario}
\subtitlesize{\HUGE}

\author{\textbf{Ricardo Cantú Calderón}}
\email{ricardocantu@ciep.mx}

%\authorb{\textbf{Juan Pablo López Reynosa}}
%\emailb{ricardocantu@ciep.mx}

%\authorc{\textbf{César Augusto Rivera De Jesús}}
%\emailc{cesarrivera@ciep.mx}

\date{\today}


\begin{document}
\maketitle
\fi

\chapter{\textbf{Presentación} al documento}

\textbf{¿Te imaginas poder proponer políticas públicas y ser parte del debate público sobre la redistribución del ingreso y la sostenibilidad fiscal en México?}

¡Con el \textbf{Simulador Fiscal \acs{CIEP}}, esto es posible! Es una herramienta digital e interactiva que te permite simular tus propias políticas fiscales en México. Te brinda una experiencia gratuita, objetiva y con información oficial para que cuantifiques, propongas y comprendas las políticas vigentes, así como sus consecuencias a corto plazo en redistribución del ingreso y a largo plazo en sostenibilidad fiscal. La información desplegada es la aprobada del Paquete Económico 2023, la reportada por la Secretaría de Hacienda y Crédito Público (SHCP) y la cuantificada por el Instituto Nacional de Estadística y Geografía (INEGI) en cuanto a producción, consumo de hogares, empleo, entre otros.

\section{Misión}

El Simulador Fiscal es tu herramienta de análisis que te permite evaluar y explorar escenarios de manera interactiva. Te ayuda a comprender cómo las políticas fiscales impactan la distribución del ingreso en nuestra sociedad y la equidad intergeneracional.

\section{Visión}

Queremos que el Simulador Fiscal sea utilizado durante las campañas presidenciales, por académicos, estudiantes, profesores y cualquier persona interesada en evaluar la factibilidad de las políticas públicas propuestas. Así, contribuimos a enriquecer el debate público ciudadano.

\section{Objetivos}

\begin{itemize}
    \item Proponer y experimentar con diferentes políticas públicas, brindándote un entorno interactivo para evaluar y explorar escenarios.
    \item Ayudarte a comprender cómo las políticas fiscales impactan la distribución del ingreso en nuestra sociedad.
    \item Evaluar la sostenibilidad fiscal desde una perspectiva ciudadana (en términos per cápita o por hogar), considerando tanto los ingresos como los gastos gubernamentales.
    \item Contribuir al enriquecimiento del debate público sobre el uso y la recaudación de los recursos públicos, fomentando una discusión informada y objetiva.
\end{itemize}

Estamos comprometidos en proporcionarte una experiencia enriquecedora, educativa y con información oficial y actualizada, que te empodere como ciudadano y promueva un diálogo informado sobre las políticas públicas.

¡Únete a nosotros y sé parte del cambio en México!

\section{Justificación}

La programación, visualización y disponibilidad del \textbf{Simulador Fiscal \acs{CIEP}} responden de manera efectiva a problemáticas relevantes en el contexto actual. Esta herramienta es importante para que la ciudadanía no se sienta ajena a las discusiones fiscales y presupuestales, y para que se puedan exigir públicamente políticas alineadas con las prioridades sociales. Las problemáticas abordadas incluyen la rendición de cuentas, transparencia proactiva en supuestos y metodologías, acceso a la información, enseñanza educativa, mejora de la gestión pública gubernamental, desarrollo y seguridad social, economía, educación, empleo, medio ambiente y/o energía, salud, equidad de género, inclusión y protección de la niñez.

El \textbf{Simulador Fiscal \acs{CIEP}} ha jugado un papel fundamental al permitir que la ciudadanía, los tomadores de decisiones, personal docente y otros actores relevantes realicen propuestas y visualicen el impacto fiscal de las mismas en temas de sostenibilidad y redistribución, con información real y actualizada. Al proporcionar una herramienta interactiva y accesible, el simulador fomenta la participación y el compromiso ciudadano en las discusiones sobre el sistema fiscal y las políticas públicas.

Es crucial destacar que el \textbf{Simulador Fiscal \acs{CIEP}} aborda las problemáticas mencionadas al permitir a las personas comprender mejor las implicaciones de las decisiones fiscales en áreas clave como el desarrollo y seguridad social, economía, educación, empleo, medio ambiente y/o energía, salud, equidad de género, inclusión y protección de la niñez. De esta manera, el simulador promueve la adopción de políticas efectivas y equitativas que abordan estas problemáticas y contribuyen a mejorar la calidad de vida de todos los ciudadanos.

El \textbf{Simulador Fiscal \acs{CIEP}} es una herramienta valiosa que empodera a la ciudadanía, permitiéndoles participar activamente en la toma de decisiones y el diseño de políticas públicas. Al abordar diversas problemáticas relevantes, fomenta la transparencia, el debate informado y la construcción de políticas fiscales orientadas a mejorar la sociedad en su conjunto.

\ifdefined\maindoc\else
\end{document}
\fi
