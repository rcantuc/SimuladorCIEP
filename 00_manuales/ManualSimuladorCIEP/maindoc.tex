%%%%%%%%%%%%%%%%%%
%%%%          %%%%
%%%% MAIN DOC %%%%
%%%%          %%%%
%%%%%%%%%%%%%%%%%%
\gdef\maindoc{1}
%%%%%%%%%%%%%%%%%%
%%%% PREAMBLE %%%%
%%%%%%%%%%%%%%%%%%
\ifdefined\maindoc
	\documentclass[acronimos,noportada,prologo]{CIEP/ciepnew}
\else
	\documentclass[acronimos,noportada]{../CIEP/ciepnew}
\fi

\title{Simulador Fiscal \acs{CIEP}}
\titlesize{\fontsize{30}{30}}

\subtitle{\bfseries Manual de usuario}
\subtitlesize{\HUGE}

\author{\textbf{Ricardo Cantú Calderón}}
\email{ricardocantu@ciep.mx}

%\authorb{\textbf{Juan Pablo López Reynosa}}
%\emailb{ricardocantu@ciep.mx}

%\authorc{\textbf{César Augusto Rivera De Jesús}}
%\emailc{cesarrivera@ciep.mx}

\date{\today}


\begin{document}
\maketitle





%%%%%%%%%%%%%%%%
%%%          %%%
%%% CONTEXTO %%%
%%%          %%%
%%%%%%%%%%%%%%%%

\hypertarget{manual-del-simulador-fiscal-ciep-v5}{%
\section{Manual del Simulador Fiscal CIEP
v5}\label{manual-del-simulador-fiscal-ciep-v5}}



\hypertarget{manual-de-uso}{%
\section{Manual de uso}\label{manual-de-uso}}

\hypertarget{presentaciuxf3n}{%
\subsubsection{Presentación}\label{presentaciuxf3n}}

El Simulador Fiscal CIEP v5 ha sido programado con la premisa de que el
Sistema Fiscal se basa en las personas y en el equilibrio presupuestario
federal. Entendemos el balance presupuestario como la diferencia entre
los ingresos presupuestarios y el gasto público federal, que puede
resultar en endeudamiento.

\[Ingresospresupuestarios - GastopúblicoFederal = Endeudamiento\]

\hypertarget{fuentes-de-informaciuxf3n}{%
\subsubsection{Fuentes de información}\label{fuentes-de-informaciuxf3n}}

La información presupuestaria se actualiza con base en los resultados
fiscales más recientes de los ingresos y el gasto público, según lo
establecido en la Ley de Ingresos de la Federación (LIF) y el
Presupuesto de Egresos de la Federación (PEF) 2023. Además, los cálculos
se realizan utilizando datos del Sistema de Cuentas Nacionales (del
Banco de Información Económica), el crecimiento demográfico del Consejo
Nacional de Población (CONAPO) y el número de personas matriculadas,
aseguradas en el sistema público de salud y pensionadas de la Encuesta
Nacional de Ingresos y Gastos de los Hogares (ENIGH) 2020. Hasta la
fecha de redacción de este manual, la información está actualizada hasta
el año 2023.

\hypertarget{por-quuxe9-es-importante}{%
\subsubsection{¿Por qué es importante?}\label{por-quuxe9-es-importante}}

El Simulador Fiscal CIEP v5 es una herramienta invaluable para abordar
los desafíos fiscales que enfrenta México. En primer lugar, el actual
sistema fiscal no cuenta con los recursos suficientes para mantener el
gasto público en el largo plazo. Aumentar la recaudación implica cambios
significativos en las prioridades públicas y no tomar medidas sería aún
más perjudicial, con grandes costos para todos los ciudadanos.

En segundo lugar, el cambio demográfico proyectado por CONAPO representa
un desafío adicional. Con el paso del tiempo, la proporción de niños
disminuye y la de adultos mayores aumenta. En 2050, se espera que haya
la misma cantidad de personas mayores de 60 años que menores de 18 años,
algo nunca antes visto. Esto implica menos personas en edad de trabajar
y más en edad de jubilarse, lo que genera una mayor presión sobre el
sistema fiscal.

Además, la deuda pública es otra preocupación significativa. Durante más
de 10 años, el gasto ha superado la recaudación, lo que ha generado un
desbalance y una creciente deuda. El costo de dicha deuda ya supera el
gasto total en salud y se espera que pronto supere incluso al de
educación. Esto limita nuestras posibilidades de inversión en sectores
clave y compromete nuestro futuro financiero.

Ante este panorama, el Simulador Fiscal CIEP v5 se vuelve esencial para
que todos podamos comprender la magnitud de los retos fiscales y
participar en su solución. Nos invita a explorar diferentes escenarios,
calcular la recaudación necesaria y proponer medidas para lograr un
sistema fiscal más equitativo y sostenible.

\hypertarget{estado-base}{%
\subsubsection{Estado base}\label{estado-base}}

Al interactuar con el Simulador Fiscal CIEP v5, se muestra el balance
presupuestario más actualizado como porcentaje (\%) del Producto Interno
Bruto (PIB). En el momento de redacción de este documento, el balance
presupuestario era de -3.8\% del PIB (un negativo represente un
endeudamiento y un positivo un ahorro), debido a que los gastos
representaron el 26.9\% del PIB, mientras que los ingresos fueron del
23.1\% del PIB (Figura 1).

Además, se presentan los diferentes conceptos que componen los ingresos,
los gastos y el marco macroeconómico nacional. Si bien estos componentes
se encuentran actualizados según el entorno económico actual, el usuario
tiene la posibilidad de modificarlos para sus propuestas y simulaciones
de política pública. En esta sección, el manual se enfoca en el estado
base y la presentación de los componentes. Para ver un ejemplo de
simulaciones, consulte la sección XXX.

Figura. 1: Balance presupuestario

%\includegraphics[width=5.90556in,height=0.58056in]{media/image4.png}

Componentes de los Ingresos

Para asegurar la claridad y accesibilidad del Simulador Fiscal CIEP v5,
es importante comprender los componentes de los ingresos presupuestarios
federales. Estos se dividen en cuatro categorías principales:
\textbf{impuestos laborales, impuestos al consumo, impuestos al
capital}\footnote{En esta sección, se incluyen los ingresos tributarios
  al capital y los ingresos no tributarios como derechos, productos y
  aprovechamientos.} \textbf{y los ingresos propios de organismos y
empresas}. Cada uno de estos componentes está vinculado a una fuente de
ingreso específica del Sistema de Cuentas Nacionales y su resultado
recaudatorio se presenta como un porcentaje del Producto Interno Bruto
(PIB), relacionado con su cuenta macroeconómica gravable (Figura 2).
Esta información proviene del Banco de Información Económica (BIE) del
INEGI. Si se quiere más información sobre las cifras presentadas, todos
los cuadros tienen su liga que te dirige a la fuente de información
primaria.

Por ejemplo, el Impuesto Sobre la Renta (ISR) se vincula a las
compensaciones de los asalariados\footnote{Excluyendo contribuciones a
  la seguridad social.}. Así, el Simulador Fiscal CIEP v5 muestra la
tasa efectiva para cada fuente de ingreso, lo que permite conocer qué
proporción de la cuenta macroeconómica gravable representaría un ingreso
público.

\[tasaefectiva\left( \text{\%} \right) = \frac{Recaudación}{Cuentamacroeconómicagravable}\]

Es recomendable tener conocimiento sobre conceptos tributarios
utilizados a lo largo de la sección de ingresos, como la tasa cero,
exenciones y tabuladores que generan los impuestos. Para ello, se
sugiere consultar el significado de dichos conceptos en la publicación
\href{https://ingresosenmexico.ciep.mx/}{``Ingresos públicos en México:
Hacia un nuevo sistema fiscal''} elaborada por CIEP en 2021. Además, se
recomienda explorar la línea de investigación sobre ingresos públicos
disponible en la página oficial de
\href{https://ciep.mx/investigaciones/}{CIEP}.

\hypertarget{impuestos-laborales}{%
\paragraph{Impuestos laborales}\label{impuestos-laborales}}

Dentro de los ingresos presupuestarios federales, los impuestos
laborales comprenden el Impuesto Sobre la Renta (ISR) aplicado a
personas asalariadas y personas físicas, así como los ingresos públicos
generados por las cuotas de seguridad social. Por un lado, estos tres
conceptos pueden ser simulados y ajustados en el simulador, modificando
su recaudación como un porcentaje del PIB (Figura 2). Por otro lado,
también es posible realizar cambios en el marco legal establecido por la
Ley del Impuesto Sobre la Renta (artículos 151, 152 y décimo
transitorio), después de dar click al bontón ``Submódulo'' (Figura 3).

Figura 2: Impuestos laborales

%\includegraphics[width=4.83125in,height=3.63056in]{media/image5.png}

Por ejemplo, en el caso del ISR sobre asalariados y personas físicas, es
posible ajustar las tasas impositivas y los subsidios aplicados a los
diferentes rangos salariales utilizados para calcular el ISR y el
subsidio al empleo. Cada impuesto está relacionado con su cuenta
macroeconómica gravable, que en este caso son las variables de ingresos
laborales. En el estado base, los impuestos laborales representan el
5.4\% del PIB y el 13.2\% (tasa efectiva) de los ingresos laborales en
México.

Además, en el simulador, es posible modificar las deducciones como un
porcentaje del ingreso sujeto a gravamen, así como el porcentaje de
informalidad (con base a la ENIGH 2020 y la recaudación reportada por la
SHCP). Para este último, se considera a la población que genera
ingresos, pero que no paga ISR (Figura 3).

Figura 3: Rangos de ingresos salariales

%\includegraphics[width=3.45833in,height=3.04306in]{media/image6.png}

\hypertarget{impuestos-al-consumo}{%
\paragraph{Impuestos al consumo}\label{impuestos-al-consumo}}

Dentro de los ingresos presupuestarios federales, los impuestos al
consumo se componen de diferentes elementos, entre ellos (Figura 4):

\begin{itemize}
\item
  Impuesto al Valor Agregado (IVA): Este impuesto se aplica a la mayoría
  de los bienes y servicios, gravando el valor agregado en cada etapa de
  producción y comercialización.
\item
  Impuesto Sobre Automóviles Nuevos (ISAN): Este impuesto se aplica a la
  adquisición de vehículos nuevos.
\item
  Impuestos sobre importaciones: Son los impuestos aplicados a los
  bienes y servicios que son importados al país.
\item
  Impuesto Especial Sobre productos y Servicios (IEPS): Este impuesto se
  aplica a productos específicos, como las gasolinas y el diésel
  (petrolero), así como a otros bienes y servicios.
\end{itemize}

Cada uno de estos impuestos está vinculado a una cuenta macroeconómica
gravable relacionada con el consumo de los hogares. En el estado base,
los impuestos al consumo representan el 6.5\% del PIB y el 10.3\% (tasa
efectiva) del valor económico del consumo de los hogares e instituciones
sin fines de lucro.

Figura 4: Impuestos al consumo

%\includegraphics[width=3.72708in,height=3.20069in]{media/image7.png}

En el simulador, es posible realizar simulaciones y ajustes relacionados
con el IVA, como la tasa cero y las exenciones aplicadas a diferentes
grupos de bienes y servicios (Figura 5). Por ejemplo, se puede eliminar
la tasa cero sobre alimentos o medicinas y evaluar su impacto
recaudatorio.

Figura 5: Gravámenes, tasa cero y exenciones del IVA

%\includegraphics[width=3.56597in,height=3.98523in]{media/image8.png}

\hypertarget{impuestos-al-capital}{%
\paragraph{Impuestos al Capital}\label{impuestos-al-capital}}

Dentro de los ingresos presupuestarios federales, los impuestos al
capital se componen de dos elementos principales (Figura 6):

\begin{itemize}
\item
  Impuesto Sobre la Renta (ISR) aplicado a personas morales: En el
  simulador, es posible realizar simulaciones sobre la tasa estatuaria
  aplicada al ISR de personas morales. En el estado base, esta tasa es
  del 30\%. Modificar esta tasa permite evaluar su impacto en los
  ingresos generados por el capital privado.
\item
  Contribuciones no tributarias: Estas contribuciones incluyen los
  productos, derechos y aprovechamientos relacionados con el capital
  privado.
\end{itemize}

Cada contribución está vinculada a una cuenta macroeconómica gravable
que representa los ingresos generados por el capital privado. En el
estado base, los impuestos al capital representan el 5.2\% del PIB y
equivalen al 16\% (tasa efectiva) del retorno del capital privado.

Figura 6: Impuestos al capital

%\includegraphics[width=5.87232in,height=3.68559in]{media/image9.png}

Además, en el simulador se puede modificar el porcentaje de personas
morales que no contribuyen a través del ISR (estimados con la ENIGH 2020
e información de la SHCP) y modificar la tasa impositiva. Esto permite
evaluar diferentes escenarios y entender cómo estos cambios impactan la
recaudación fiscal por parte del capital privado (Figura 7).

Figura 7: Parámetros de recaudación sobre personas morales

%\includegraphics[width=5.90556in,height=2.58681in]{media/image10.png}

\hypertarget{organismos-y-empresas}{%
\paragraph{Organismos y empresas}\label{organismos-y-empresas}}

Dentro de los ingresos presupuestarios federales, los ingresos de
organismos y empresas públicas abarcan distintas entidades, entre las
cuales se encuentran (Figura 8):

\begin{itemize}
\item
  Empresas presupuestarias de control directo: Estas empresas incluyen
  instituciones como el Instituto Mexicano del Seguro Social (IMSS) y el
  Instituto de Seguridad y Servicios Sociales de los Trabajadores del
  Estado (ISSSTE).
\item
  Empresas productivas del Estado: Este grupo incluye entidades como la
  Comisión Federal de Electricidad (CFE) y Petróleos Mexicanos (Pemex).
\item
  Transferencias al Fondo Mexicano del Petróleo (FMP): Estas
  transferencias también forman parte de los ingresos de organismos y
  empresas públicas.
\end{itemize}

\begin{quote}
Figura 8: Ingresos de organismos y empresas
\end{quote}

%\includegraphics[width=4.2793in,height=3.72437in]{media/image11.png}

Cada uno de estos componentes está vinculado a la cuenta macroeconómica
gravable del ingreso de capital público y contribuye al total
recaudatorio de ingresos de capital. Mientras que, el total se vincula
con los ingresos de capital, pero descontando el consumo de capital
fijo. En el estado base, estos ingresos representan el 5.8\% del PIB y
el 15.2\% (tasa efectiva) del ingreso de capital.

\hypertarget{componentes-del-gasto-puxfablico}{%
\subsubsection{Componentes del Gasto
público}\label{componentes-del-gasto-puxfablico}}

El gasto público federal se divide en cinco categorías principales:
educación pública, salud, pensiones, otros gastos y un posible ingreso
básico universal. Cada una de estas categorías está vinculada a su
población beneficiaria y se relaciona con el resultado recaudatorio
actualizado como porcentaje del Producto Interno Bruto (PIB) y el gasto
per cápita. Las simulaciones se realizan con base en el gasto per
cápita.

A lo largo de la sección de gasto, se utilizan conceptos de gasto con
base en las clasificaciones del presupuesto; por lo que, se recomienda
la línea de investigación sobre el gasto público disponible en la página
oficial de \href{https://ciep.mx/investigaciones/}{CIEP}.

\hypertarget{educaciuxf3n-puxfablica}{%
\paragraph{Educación Pública}\label{educaciuxf3n-puxfablica}}

El gasto educativo abarca desde la educación inicial hasta el posgrado,
incluyendo también la educación para adultos y otros gastos educativos.
En el simulador, es posible ajustar el gasto per cápita en educación. En
el estado base, el gasto educativo total equivale al 3.3\% del PIB y a
32,103 pesos por alumno (Figura 9).

Figura 9: Gasto en educación pública

%\includegraphics[width=4.64931in,height=3.80903in]{media/image12.png}

\hypertarget{salud-puxfablica}{%
\paragraph{Salud pública}\label{salud-puxfablica}}

El gasto en salud se compone de distintas entidades públicas que brindan
servicios de salud, como IMSS Bienestar, ISSSTE, Pemex y el Instituto de
Seguridad Social para las Fuerzas Armadas Mexicanas (ISSFAM). Además, se
considera el gasto en salud para la población no afiliada a estos
servicios públicos. Al igual que en otros componentes, es posible
ajustar el gasto per cápita en salud. En el estado base, el gasto en
salud equivale al 2.9\% del PIB y a 6,804 pesos por persona (Figura 10).

Figura 10: Gasto en salud

%\includegraphics[width=4.57733in,height=3.34319in]{media/image13.png}

\hypertarget{pensiones}{%
\paragraph{Pensiones}\label{pensiones}}

El gasto en pensiones incluye distintas entidades públicas responsables
del pago de pensiones, como IMSS, ISSSTE, Pemex, CFE, ISSFAM, Ferronales
y la extinta Luz y Fuerza del Centro (LFC). También se contempla el
gasto por la pensión del bienestar para adultos mayores. Al igual que en
los otros componentes, se puede ajustar el gasto per cápita en
pensiones. En el estado base, el gasto en pensiones equivale al 5.5\%
del PIB y a 93,256 pesos por persona (Figura 11).

Figura 11: Gasto en Pensiones

%\includegraphics[width=5.90556in,height=3.97569in]{media/image14.png}

\hypertarget{otros-gastos}{%
\paragraph{Otros gastos}\label{otros-gastos}}

Los otros gastos corresponden a distintas entidades públicas
asignatarias del gasto público, como CFE, Pemex, Secretaría de Energía
(SENER) y otros gastos energéticos, infraestructura, costo de la deuda,
gasto federalizado y el resto de los gastos. En este caso, el gasto per
cápita se calcula en función de la población total. En el estado base,
estos gastos representan el 15.2\% del PIB y 35,779 pesos por persona
(Figura 12).

Figura 12: Otros gastos

%\includegraphics[width=3.31171in,height=2.93241in]{media/image15.png}

\hypertarget{ingreso-buxe1sico}{%
\paragraph{\texorpdfstring{ Ingreso
básico}{ Ingreso básico}}\label{ingreso-buxe1sico}}

Por último, el simulador permite explorar un posible ingreso básico para
toda la población. Es posible ajustar las opciones para excluir a la
población menor de edad o a la población adulta mayor. Por ejemplo, se
puede asignar un ingreso básico a toda la población marcando ambas
opciones disponibles. También se pueden marcar únicamente el ingreso
básico a mayores de 65 años o a menores de 18 años. En resumen, las
personas entre 18 y 68 años siempre recibirían el ingreso básico (Figura
13).

Figura 13: Ingreso básico

%\includegraphics[width=5.90556in,height=2.56111in]{media/image16.png}

\hypertarget{marco-macroeconuxf3mico}{%
\subsubsection{Marco macroeconómico}\label{marco-macroeconuxf3mico}}

En el marco macroeconómico se toman en cuenta las estimaciones
proporcionadas por la Secretaría de Hacienda en los Criterios Generales
de Política Económica, presentados como parte del Paquete Económico.
Estas estimaciones incluyen el crecimiento proyectado del Producto
Interno Bruto (PIB) y la inflación para el año fiscal en curso y los
años venideros.

Además, se consideran estimaciones relacionadas con los intereses de la
deuda pública, expresados como porcentaje del Saldo Histórico de los
Requerimientos Financieros del Sector Público (SHRFSP). También se
incluyen datos sobre el tipo de cambio frente al dólar y la depreciación
esperada del peso mexicano respecto al dólar estadounidense. Estos
factores tienen un impacto significativo en la economía y se reflejan en
las simulaciones realizadas en el simulador (Figura 14).

Figura 14: Marco macroeconómico

%\includegraphics[width=2.93472in,height=4.53681in]{media/image17.png}

\hypertarget{resultados}{%
\section{Resultados}\label{resultados}}

Después de seleccionar los parámetros disponibles en ingresos, gastos y
marco macroeconómico, el Simulador Fiscal CIEP v5 proporciona resultados
que pueden ser visualizados en diferentes aspectos. Estos incluyen la
incidencia de las políticas fiscales por deciles de ingreso, grupos
etarios y género; estimaciones futuras de ingresos, gastos y deuda
pública; así como la redistribución del sistema fiscal a través del
ciclo fiscal.

Es importante destacar que el Simulador Fiscal CIEP v5 está diseñado
como un equilibrio parcial, también conocido como "aritmético". Esto
implica que las variables simuladas no afectan a otros parámetros en el
modelo. Por ejemplo, si la simulación incorpora aumentos en el gasto
público en educación, salud e infraestructura, esto no tendría un
impacto simulado en el crecimiento económico o en los ingresos futuros
de los hogares, entre otros aspectos.

Nota adicional: El enfoque del Simulador Fiscal CIEP v5 se centra en el
análisis de políticas fiscales y sus efectos directos en los indicadores
seleccionados. Sin embargo, es importante tener en cuenta que existen
otros factores y variables macroeconómicas que pueden influir en los
resultados generales de la economía, pero que no están contemplados en
esta herramienta de simulación específica. Por tanto, es recomendable
considerar estos resultados como una visión parcial y complementar el
análisis con información adicional.

\hypertarget{incidencia-por-deciles}{%
\subsubsection[Incidencia por deciles]{\texorpdfstring{Incidencia por
deciles\footnote{Cada decil representa 10\% de la población ordenado por
  nivel de ingreso. El primero decil representaría al promedio del 10\%
  con menos ingresos; mientras que, el décimo representaría al 10\% con
  mayores ingresos.}}{Incidencia por deciles}}\label{incidencia-por-deciles}}

Uno de los resultados visibles en el Simulador Fiscal CIEP v5 es la
distribución de las aportaciones netas de los hogares al sistema fiscal,
mediante el uso de la ENIGH 2020. Se considera como aportación neta la
diferencia entre los ingresos y el gasto públicos aportado y recibido
por cada hogar. Con el fin de analizar la progresividad del sistema
fiscal, los resultados se segmentan por deciles. Además, se presentan
las aportaciones netas promedio por hogar, la distribución porcentual de
estas aportaciones y el promedio de cuánto representan las aportaciones
netas en relación con el ingreso anual del hogar (Figura 15).

Figura 15: Incidencia por deciles

%\includegraphics[width=5.90556in,height=3.37847in]{media/image18.png}

En el estado base, el 10\% de los hogares con menores ingresos presenta
una aportación neta negativa de 57,313 pesos. Esto implica que reciben
más gasto público del que aportan por impuestos y las aportaciones netas
representan, en promedio, el 73.2\% de su ingreso anual. Por otro lado,
el 10\% de los hogares con mayores ingresos realiza una aportación neta
promedio de 285,583 pesos, que representa el 14.5\% de su ingreso anual.

Nota adicional: Los resultados presentados son una descripción general
de la incidencia por deciles en el estado base. La progresividad del
sistema fiscal puede variar según las políticas fiscales y los
parámetros seleccionados en el simulador. Es importante analizar estos
resultados en conjunto con otras variables y considerar que pueden
existir otras formas de medir la progresividad y equidad del sistema
fiscal.

\hypertarget{incidencia-por-edades}{%
\subsubsection{Incidencia por edades}\label{incidencia-por-edades}}

En concordancia con la importancia del ciclo de vida fiscal, se
presentan resultados diferenciados por sexo, grupos de deciles y rangos
de edad, también mediante el uso de la ENIGH 2020. De esta manera, se
puede observar cómo las personas contribuyen a lo largo de su vida
(Figura. 16).

Como resultado general, se observa que las niñas, niños y adolescentes
(NNA) y las personas adultas mayores tienden a recibir más del sistema
fiscal de lo que aportan. Esto se debe a que, al no estar en edad de
trabajar, suelen ser beneficiarios de programas sociales y contribuyen a
través de impuestos al consumo. Por otro lado, cuando una persona se
encuentra en edad laboral, tiende a contribuir al sistema fiscal
mediante impuestos laborales, de capital y consumo, mientras que suelen
recibir menos gasto fiscal en relación con sus contribuciones. Además,
los roles de género, como las diferencias salariales y de oportunidades,
entre otros factores, se reflejan en los resultados observados.

Es importante destacar que estos resultados evidencian patrones
generales y no representan necesariamente la situación individual de
cada persona. La forma en que cada individuo contribuye y se beneficia
del sistema fiscal puede variar según su situación personal, ingresos,
gastos y otros factores específicos. Por lo tanto, es fundamental
considerar estos resultados en un contexto más amplio y complementarlos
con análisis adicionales para tener una visión completa de la equidad y
efectividad del sistema fiscal.

Figura 16: Incidencia por edades

%\includegraphics[width=5.90556in,height=4.17083in]{media/image19.png}

\hypertarget{proyecciuxf3n-de-las-aportaciones}{%
\subsubsection{Proyección de las
aportaciones}\label{proyecciuxf3n-de-las-aportaciones}}

Dada la existente y difícil de revertir transición demográfica, el
CONAPO ha proyectado una tendencia en la que cada vez hay menos personas
en edad de trabajar y, en contraste, un aumento en la demanda de
servicios públicos. Teniendo en cuenta las aportaciones netas
mencionadas anteriormente y el contexto demográfico, se estima que el
punto máximo de estas contribuciones se alcance en 2025, a partir del
cual podrían comenzar a disminuir (Figura 17). Es decir, a partir de
entonces, la población empezará a demandar más servicios por pensiones y
salud y contribuirá menos por impuestos.

Con el fin de retrasar el punto máximo de las aportaciones netas, se
sugieren algunas recomendaciones de simulación. Dado que las personas
que no se encuentran en edad laboral, como los NNA y las personas
adultas mayores, contribuyen al sistema fiscal a través del consumo, una
manera de retrasar este punto sería mediante la implementación de
mayores impuestos al consumo. Otra estrategia sería la reducción de
programas sociales que benefician a esta población.

Figura 17: Proyección de las aportaciones

%\includegraphics[width=5.90556in,height=2.625in]{media/image20.png}

\hypertarget{ingresos-futuros}{%
\subsubsection{Ingresos futuros}\label{ingresos-futuros}}

En la estimación de los ingresos públicos futuros (Figura 18), se toman
en cuenta varios factores, como los componentes fiscales de la
recaudación presupuestaria, las tasas efectivas, los componentes del
PIB, el crecimiento económico y la transición demográfica. Los
resultados se presentan desglosados según los ingresos derivados de
impuestos al consumo, impuestos laborales y de capital, tanto públicos
como privados.

Como recomendación de simulación, es importante recordar que, para
mantener un equilibrio presupuestario, los ingresos deben ser mayores
que los gastos. Ante los cambios demográficos, es fundamental considerar
la población contribuyente de cada tipo de ingreso al realizar
simulaciones. Por ejemplo, si se estima que habrá menos personas en edad
de trabajar en el futuro, se podría considerar incrementar los impuestos
al consumo como una forma de aumentar los ingresos futuros.

Figura 18: Ingresos públicos futuros

%\includegraphics[width=5.90556in,height=2.76667in]{media/image21.png}

\hypertarget{gastos-futuros}{%
\subsubsection{Gastos futuros}\label{gastos-futuros}}

La estimación futura del gasto público (Figura 19) se realiza
considerando diversos componentes fiscales del presupuesto, la
composición demográfica y el costo de la deuda. Se proyectan los
resultados en el tiempo a partir de los gastos per cápita estimados. Los
resultados se desagregan por las áreas de educación, pensiones, salud,
costo de la deuda, pensión del bienestar y renta básica.

Es fundamental tener en cuenta que, para mantener un balance
presupuestario, los gastos deben ser menores a los ingresos. Ante los
cambios demográficos, es necesario considerar la población beneficiaria
de cada concepto de gasto. Es importante tener presente que, si la
simulación genera un desbalance presupuestario, esto puede resultar en
un aumento en el costo de la deuda.

Por ejemplo, si se estiman mayores gastos per cápita en pensiones y se
presenta un desbalance presupuestario, el gasto en pensiones y el costo
de la deuda pueden reducir el presupuesto disponible para otros
conceptos, como educación y salud. Por lo tanto, es necesario evaluar
cuidadosamente las prioridades de gasto y considerar los compromisos
financieros a largo plazo.

Figura 19: Gasto público futuro

%\includegraphics[width=5.90556in,height=2.62708in]{media/image22.png}

\hypertarget{deuda}{%
\subsubsection{Deuda}\label{deuda}}

El (des)balance presupuestario, junto con el SHRFSP actual y
considerando el crecimiento económico, entre otras variables, da lugar
al SHRFSP futuro (Figura 20). Este saldo representa las obligaciones
fiscales que se heredan a las generaciones futuras, lo que permite
visualizar la sostenibilidad y, en cierta medida, la desigualdad fiscal
intergeneracional.

Es importante tener en cuenta que el SHRFSP futuro refleja la
acumulación de deuda y compromisos financieros que afectarán las
finanzas públicas en el futuro. Un balance presupuestario positivo
contribuye a mantener una trayectoria sostenible de la deuda, mientras
que un balance negativo puede resultar en un aumento de la carga de la
deuda y generar preocupaciones sobre la capacidad del gobierno para
cumplir con sus obligaciones.

En la simulación, se recomienda realizar escenarios con balances
presupuestarios positivos y negativos para observar el comportamiento
del SHRFSP futuro. Además, es importante considerar diferentes
estimaciones de crecimiento del PIB, ya que este factor influye en la
relación entre el comportamiento de la deuda.

El Simulador Fiscal CIEP v5 ofrece una herramienta útil para evaluar y
analizar el impacto de diferentes escenarios presupuestarios en la
sostenibilidad de la deuda y en la equidad intergeneracional. Sin
embargo, es fundamental tener en cuenta que las proyecciones y
resultados son estimaciones basadas en supuestos y variables, y están
sujetas a incertidumbre y cambios en el entorno económico y político. Se
recomienda realizar análisis detallados y considerar múltiples factores
al interpretar y utilizar los resultados de la simulación.

Figura 20: Proyección de la deuda

%\includegraphics[width=5.90556in,height=2.58889in]{media/image23.png}

\hypertarget{redistribuciuxf3n-fiscal}{%
\subsubsection{Redistribución fiscal}\label{redistribuciuxf3n-fiscal}}

Por último, es importante destacar que el sistema fiscal comienza y
termina en las personas. El Simulador Fiscal CIEP v5 permite visualizar
este flujo fiscal mediante un diagrama de Sankey (Figura. 21),
desglosando las aportaciones fiscales de los diferentes deciles de la
población y mostrando cómo se distribuye el gasto público entre ellos.

Al utilizar el Simulador Fiscal CIEP v5, se puede observar claramente
cómo los diferentes deciles contribuyen al ingreso presupuestario a
través de sus aportaciones fiscales, y cómo reciben el gasto público en
forma de beneficios y servicios. El diagrama de Sankey proporciona una
representación visual efectiva de esta interacción entre los deciles y
el sistema fiscal. Además, el simulador ofrece la capacidad de aislar un
tipo específico de ingreso o gasto, estableciendo los demás parámetros
en cero. Esto permite analizar el impacto de políticas públicas
individuales y observar cómo afectan a los diferentes deciles de la
población.

Recuerda que la simulación es una herramienta valiosa para comprender
las implicaciones de las decisiones fiscales y evaluar distintos
escenarios. Al explorar y analizar los resultados del Simulador Fiscal
V5, se puede obtener una mejor comprensión de cómo las políticas
fiscales afectan a las personas y cómo se distribuyen los recursos a lo
largo de los diferentes deciles de la sociedad.

¡Sigue utilizando el Simulador Fiscal y explora los resultados
detallados para obtener una visión más clara de cómo el sistema fiscal
impacta a cada segmento de la población y cómo se pueden diseñar
políticas más equitativas y eficientes!

Figura 21: Diagrama de Sankey

%\includegraphics[width=5.90556in,height=3.21806in]{media/image24.png}

\hypertarget{ejemplos-de-simulaciuxf3n}{%
\section{Ejemplos de simulación}\label{ejemplos-de-simulaciuxf3n}}

En esta sección, llevaremos a cabo una simulación y análisis de
resultados utilizando el Simulador Fiscal CIEP v5. El objetivo es
proporcionar un ejemplo práctico sobre cómo utilizar los diferentes
parámetros y cómo interpretar los resultados obtenidos. Es importante
destacar que no existen limitaciones para la imaginación en este
simulador. Puedes simular políticas públicas basadas en la literatura
disponible o simplemente dejarte llevar por tu propia curiosidad. La
herramienta te brinda la libertad de explorar y experimentar diferentes
escenarios fiscales para comprender mejor su impacto.

\hypertarget{descripciuxf3n-de-la-simulaciuxf3n}{%
\subsubsection{Descripción de la
simulación}\label{descripciuxf3n-de-la-simulaciuxf3n}}

En esta simulación, nos enfocaremos en dos aspectos clave: los ingresos
y los gastos. Por el lado de los ingresos, se implementará un gravamen
de IVA sobre alimentos y medicinas utilizando los módulos de impuestos
al consumo en IVA (Figura 22). Por otro lado, en cuanto a los gastos, se
aplicará un ingreso básico para la población menor de 65 años (Figura
23).

Es importante tener en cuenta que esta simulación es ilustrativa y no
debe interpretarse como una recomendación de política pública. Su
propósito es brindar un ejemplo práctico y permitirnos explorar las
implicaciones de diferentes escenarios fiscales.

\hypertarget{balance-presupuestario}{%
\subsubsection{Balance presupuestario}\label{balance-presupuestario}}

%\includegraphics[width=3.33889in,height=4.53889in]{media/image25.png}%\includegraphics[width=3.33889in,height=4.53889in]{media/image26.png}En
cuanto al balance presupuestario, los resultados iniciales revelan que
la recaudación por IVA representaría el 6.2\% del PIB (Figura 22), lo
cual es 1.6\% del PIB más alto en comparación con el estado base, que es
del 4.6\%. Por otro lado, el gasto en la pensión universal sería del
1.5\% (Figura 23), mientras que en el estado base no había asignación
para este rubro. Estos mayores ingresos contribuirían a una disminución
en el endeudamiento, pasando del 3.81\% del PIB al 3.78\%. Además, la
estimación de la deuda per cápita en 2030 sería de 181,492 pesos, una
cifra ligeramente menor que la del estado base, que es de 182,152 pesos
(Figura 24). Aunque estos cambios no tendrían un impacto sustancial en
el balance presupuestario, se observan resultados interesantes en otras
visualizaciones.

%\includegraphics[width=3.72078in,height=1.76623in]{media/image29.png}%\includegraphics[width=6in,height=0.96824in]{media/image31.png}%\includegraphics[width=6in,height=0.96824in]{media/image32.png}

\hypertarget{incidencia-por-deciles-1}{%
\subsubsection{Incidencia por deciles}\label{incidencia-por-deciles-1}}

Analizando la distribución de las aportaciones netas por deciles de los
hogares (Figura 25), se observa que el 10\% de menores ingresos tendría
una aportación negativa de 67 mil 383 pesos (86\% del ingreso del hogar)
en comparación con los 57 mil 413 pesos (73.2\% del ingreso del hogar)
en el estado base. Esto indica que, a pesar de pagar más IVA, las
familias del primer decil recibirían relativamente más a través del
ingreso básico. Este resultado es similar para los deciles I al V, lo
que significa que el 50\% de los hogares de menores ingresos se
beneficiaría de esta medida.

En contraste, el 10\% de los hogares con mayores ingresos aportaría en
promedio 300 mil 579 pesos (11.9\% del ingreso del hogar) en comparación
con los 285 mil 583 pesos (14.5\% del ingreso del hogar) en el estado
base. Por lo tanto, el pago de IVA de estos hogares sería mayor que el
ingreso básico recibido. Este patrón se observa de manera similar en los
deciles VI al X.

Figura 23: Incidencia por deciles

%\includegraphics[width=5.84348in,height=2.24348in]{media/image35.png}%\includegraphics[width=5.84348in,height=2.24348in]{media/image36.png}

\hypertarget{incidencia-por-edades-y-sexo}{%
\subsubsection{Incidencia por edades y
sexo}\label{incidencia-por-edades-y-sexo}}

En cuanto a los resultados por edad y sexo (Figura 24), no se observan
cambios visualmente perceptibles en las tendencias generales. Según la
simulación, los hombres continuarían realizando mayores aportaciones al
sistema fiscal en comparación con las mujeres, y la etapa laboral
seguiría siendo la principal en términos de aportaciones netas.

Sin embargo, al explorar detalladamente la página del Simulador Fiscal
CIEP v5, se pueden apreciar cambios marginales en puntos específicos de
la gráfica. Por ejemplo, los hombres de entre 40 y 44 años aumentarían
su aportación al 0.38\% del PIB, cifra superior al 0.36\% aportado en el
estado base. Este patrón puede verificarse de manera similar para cada
grupo de edad y por grupos de deciles.

Figura 24:Incidencia por edades

%\includegraphics[width=4.33913in,height=2.91304in]{media/image39.png}%\includegraphics[width=4.33913in,height=2.91304in]{media/image40.png}

\hypertarget{ingresos-y-gastos-futuros}{%
\subsubsection{Ingresos y Gastos
futuros}\label{ingresos-y-gastos-futuros}}

Para proyectar los ingresos y gastos futuros, nos basaremos en el cuadro
auxiliar disponible en la versión en línea del Simulador Fiscal V5.
Según las estimaciones, para el año 2030 se prevé que los ingresos por
consumo alcancen los 2.602 billones de pesos, en comparación con los
2.087 billones de pesos del estado base. Esto indica que con el tiempo,
los ingresos por IVA aumentarían, generando aproximadamente 1.25
billones de pesos adicionales en recaudación\footnote{Los cambios en las
  demás cifras se deben al redondeo de los números; por lo que
  recomendamos no considerar los resultados ajenos a tu simulación. Ya
  que pueden tener una desviación original respecto al estado base.}
(Figura 25).

De manera similar, se estima que el gasto extraordinario por el ingreso
básico para 2030 sería de 0.494 billones de pesos, cantidad menor a los
1.25 billones de pesos adicionales generados por el gravamen al IVA.
Además, al observar un déficit presupuestario menor al estado base, se
espera que el endeudamiento hasta 2030 sea ligeramente inferior, lo que
a su vez reduciría el costo de la deuda. Según las proyecciones, el
costo de la deuda estimado para 2030 sería de 1.58 billones de pesos,
una cifra ligeramente menor en .006 billones de pesos en comparación con
el estado base. Aunque este monto es relativamente pequeño en
comparación con otros rubros, este resultado ilustra el efecto de
reducir el déficit presupuestario a lo largo del tiempo en el costo de
la deuda (Figura 26).

Figura 25: Ingresos futuros

%\includegraphics[width=4.02609in,height=4.74783in]{media/image43.png}%\includegraphics[width=4.02609in,height=4.74783in]{media/image44.png}

%\includegraphics[width=4.80486in,height=6.20764in]{media/image47.png}%\includegraphics[width=4.80486in,height=6.20764in]{media/image48.png}

\hypertarget{redistribuciuxf3n-del-sistema-fiscal}{%
\subsubsection{Redistribución del sistema
fiscal}\label{redistribuciuxf3n-del-sistema-fiscal}}

Por último, en el diagrama de Sankey se muestra el gasto correspondiente
al ingreso básico propuesto en la simulación. Este gasto representa el
1.5\% del PIB y se distribuye de la siguiente manera: el 0.16\% para los
deciles I a IV, y el 0.15\% para cada decil a partir del V.

En cuanto a los impuestos al consumo, el décimo decil aportaría el
1.99\% del PIB, mientras que en el estado base esa cifra era del 1.61\%
del PIB. En contraste, el primer decil aportaría el 0.24\% del PIB, en
comparación con el 0.19\% del estado base. La diferencia en las
contribuciones del primer decil sería de aproximadamente 0.05\% del PIB,
lo cual es mayor en comparación con la pensión universal. Por lo tanto,
el primer decil se beneficiaría más de esta medida. Esta comparación se
puede realizar para todos los deciles.

Figura 26: Diagrama de Sankey (simulación)

%\includegraphics[width=6.16114in,height=3.34441in]{media/image51.png}

Querido lectora o lector,

Espero que esta conversación haya sido de utilidad para ti y que hayas
encontrado en nuestro manual una herramienta valiosa para comprender y
simular diferentes escenarios en las finanzas públicas. Nuestro objetivo
principal ha sido proporcionarte un ejemplo práctico de cómo utilizar
parámetros relacionados al sistema fiscal y analizar distintos
resultados, todo ello gracias al Simulador Fiscal CIEP v5.

La relevancia de este manual radica en su capacidad para democratizar el
conocimiento sobre las finanzas públicas y abrir nuevas puertas en el
campo de las políticas públicas. No hay restricciones para tu
imaginación, puedes simular desde políticas basadas en la literatura
existente hasta ideas surgidas de tu propia curiosidad. La simulación
nos permite explorar diversas perspectivas y entender mejor las
implicaciones de nuestras decisiones en el ámbito de las finanzas
públicas.

Al democratizar las finanzas públicas, estamos fomentando un mayor
involucramiento de la sociedad en los asuntos que nos conciernen a
todos. Este manual no solo ofrece información, sino que también busca
generar conocimiento y promover un diálogo constructivo en la búsqueda
de soluciones y mejoras para nuestro entorno público.

Recuerda que los resultados aquí presentados son ilustrativos y no
representan recomendaciones de política pública. Sin embargo, a través
de las visualizaciones y datos proporcionados, podemos identificar
interesantes hallazgos y reflexionar sobre el impacto de nuestras
decisiones en el ámbito de las finanzas públicas.

¡Te animo a seguir explorando y utilizando el Simulador Fiscal CIEP v5
para expandir tu conocimiento y participar activamente en el debate
sobre las políticas públicas!

Hasta pronto,

Equipo CIEP.

%\includegraphics[width=8.46429in,height=10.65873in]{media/image52.png}




\end{document}